\documentclass{beamer}
\useoutertheme[subsection=false]{miniframes}
\setbeamertemplate{navigation symbols}{}

\usepackage{minted}

\title{Using Rust in RSE}
\author{Alex H. Room}
\date{2025/10/29}

\begin{document}
\begin{frame}
\titlepage
\end{frame}

\begin{frame}{Introduction}
  \begin{itemize}
    \item Experiences using Rust for pySpinW
    \item Binding Python to Rust
    \item Linear algebra
    \item Parallelism
  \end{itemize}
\end{frame}

\begin{frame}{How are we using it for pySpinW?}
  \begin{itemize}
    \item Core spin-wave calculation
    \item Mostly small/medium matrix linear algebra
    \item In parallel over $\mathbf{Q}$ points
  \end{itemize}
\end{frame}

\begin{frame}{The headlines}
  \begin{itemize}
    \item 50x-100x speedup over NumPy
    \item Writing Rust code is very quick (once you're over the learning curve)
    \item Paralellism is so, so easy
    \item Linear algebra is a underdeveloped but great tools exist
    \item Python bindings are a little harder...
  \end{itemize}
\end{frame}

\begin{frame}{Benchmarks}
  \begin{tabular}{l l l l l l}
    Language & & & & & \\
    \hline\hline
    Python (\texttt{NumPy}) & 2.37 & 2.02 & 4.76 & 4.11 & 312.74 \\
    Rust (\texttt{faer}) & 0.07 & 0.08 & 0.17 & 0.20 & 3.59 \\
  \end{tabular}
\end{frame}

\begin{frame}{The learning curve}
  \begin{itemize}
    \item Rust is weird if you're used to Python/C/C++
    \item Some of that is new weirdness, some is just 40 years of programming language evolution
    \item You're forced to get good straight away, sloppy code isn't allowed
    \item Code either runs safely or doesn't compile!
    \item \textit{Programming Rust} by Blandy, Orendorff and Tindall(!!)
  \end{itemize}
\end{frame}

\begin{frame}[fragile]{Parallelism with \texttt{rayon}}
  \begin{minted}{rust}


    fn my_function(inputs: Vec<f64>) -> Vec<f64> {
      inputs.into_iter().map(do_something).collect(); 
    }

    fn do_something(input: f64) -> f64 {
      // Some expensive calculation
    }
  \end{minted}
\end{frame}

\begin{frame}[fragile]{Parallelism with \texttt{rayon}}
  \begin{minted}{rust}
    use rayon::prelude::*;

    fn my_function(inputs: Vec<f64>) -> Vec<f64> {
      inputs.into_par_iter().map(do_something).collect(); 
    }

    fn do_something(input: f64) -> f64 {
      // Some expensive calculation
    }
  \end{minted}
\end{frame}

\begin{frame}{Linear algebra in Rust}
  \begin{itemize}
    \item Lots of options, none as mature as \texttt{NumPy}/\texttt{SciPy}
    \item \texttt{nalgebra} - 1D/2D arrays, decently fast, some issues... 
    \item \texttt{ndarray} - similar to NumPy, general arrays, slow!
    \item \texttt{faer} - focused on medium/large matrix algebra, very fast, easiest to use
  \end{itemize}
\end{frame}

\begin{frame}{Python bindings with \texttt{pyo3}}
  \begin{itemize}
    \item \texttt{pyo3} is the main way to bind Python to Rust
    \item Best used with \texttt{maturin} to build packages!
    \item Most basic Python types supported out of the box
  \end{itemize}
\end{frame}

\begin{frame}{Caveat emptor}
  \begin{itemize}
    \item Memory management between Python and Rust is tricky
    \item \texttt{pyo3} forbids some Rust functionality to manage this
    \item Numpy arrays need special handling which can be unintuitive
  \end{itemize}
\end{frame}

\begin{frame}{Python bindings with \texttt{pyo3}}
  Let's get hands on!
\end{frame}

\end{document}
